\chapter{Future Work}

Based on the positive feedback that this work has received, I am confident that a
more polished and feature-rich system will inspire a similar Crisis Text Line project
to be deployed in the future. This is the main goal because we want counselors to
make use of our system in their counseling sessions. The first step towards that goal
would be to implement the constructive elements suggested by users during testing.

The addition of counselor notes to the system would simply require website
functionality that allows for the input and exhibition of notes in an organized manner.
As for an emphasized display of important topics such as suicidal thoughts or high
risk issues, additional topic modeling techniques would have to be used for the
detection of these specific topics in the corpus of words. However, I believe this feature
is definitely worth implementing for better client support, and it is possible in terms
of the technical machine learning work. Topic models for specialized detection could
also be used to extract specific details from conversations such as the people involved
or relevant locations and times.

One idea to enhance the system's performance for developing an action plan is to
provide topic-based resources to counselors as they are talking with clients. Although
this strategy is controversial in terms of whether it is helpful or distracting during
an ongoing conversation, specialized resources such as training slides for how to deal
with a specific issue might be beneficial to counselors with less experience. These
resources would have to be compiled and tested by crisis support centers.

To explore more use cases for the existing system, I would first design more user
evaluations covering situations such as context switching and counselor training.
Context switching was not included in our evaluation this time because the contextual
inquiry showed more significant pain points in other areas. Another area that was
not covered in this thesis is counselor training. Training can play an important role
in the quality of counseling, especially for new volunteer counselors. Supervisors can
utilize our visual indexing feature to find helpful instances of client messages related
to specific topics. Then a counselor can be easily trained for a specific area, such as
crises involving jobs or family.

Hypotheses about new use cases would of course need to be evaluated. With
more resources, we could also try less subjective evaluation methods, such as time
measurements for the completion of a task or brain-related experiments to measure
cognitive load. Any amount of additional user testing could improve the system by
potentially eliminating any of the four visualizations that do not seem to be helpful
or necessary.

Other features that could be useful for counselors or their supervisors include
visualization of a client's history across all of their conversations and the ability to
search conversations by topic. These suggestions are more aimed toward finding
clarifying patterns between conversations of a repeat client or between conversations
with the same topics.

Finally, I believe our visual indexing work is very generalizable to other uses
outside of the mental health area. For example, the same type of topic model
visualizations could be used as a filtering mechanism to find specific topic points in
customer service conversations. It would be interesting to see how broadly this line
of work applies.

I have only listed some of my ideas for future work here, but this research area at the
intersection of mental health, topic modeling, and visualization work is so new that
there are countless directions to explore.
