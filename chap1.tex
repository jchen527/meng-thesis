\chapter{Introduction}

Crisis Text Line (CTL) is an organization that provides counseling services to young
people in crisis through texting. The goal of this thesis is to supply CTL counselors
with assistive tools in order to offer their clients the best possible service. This section
explains the motivation behind this research project, the problems we want to solve,
the approaches to implement and evaluate, and the contributions made.

\section{Motivations for Mental Health Visualizations}

The main motivating factor for this thesis is to help people with mental health crises.
Many people suffer from depression, suicidal thoughts, and emotional stress every
day. Crisis Text Line provides an outlet for clients to discuss their issues and ask
for support. However, there is a shortage of counselors compared to the
number of people seeking aid. Each counselor may have to manage various
conversations continuously for several hours. Counselors spend time on extraneous
things such as reports. In addition to maximizing counseling time for clients,
we want to assist counselors
because crisis counselors tend to have a high attrition rate due to burnout and low morale.
Many of these counselors are simply volunteers who undergo a short period of training.
These constraints motivate us to maximize the amount of time and brainpower counselors
spend on client support. Time is a critical factor in mental health situations
because clients may be at risk of suicide or physical harm.

Fortunately, the unique thing about a texting hotline is that the use of written
communication allows computer programs to analyze and extract meaningful data
from the text. Topic modeling is a machine learning technique that discovers abstract
topics occurring in a set of documents. It can be useful for summarizing large amounts
of text. Counselors may benefit from summaries of their various conversations with
clients, but topic modeling is a complex and advanced artificial intelligence concept.
Therefore, we would like to provide counselors with an easy method of understanding
the data through visualizations. Visualization is a powerful approach for presenting
data that most people are familiar with.

\section{Problem Definition}

As mentioned in the motivation section, some main difficulties with Crisis Text Line
are that there are not enough counselors to talk with all of the clients in crisis, the
counselors are usually context switching between multiple different client texts at a
time, they spend a nontrivial amount of time on other necessary tasks, and
counselors may feel burned out. Since it is
more difficult to control external factors such as the number of available counselors,
we focus on tackling two specific problems:
\begin{enumerate}
  \item Reduce the amount of time counselors spend not talking to clients.
  \item Reduce the cognitive load of counselors so they feel less burned out.
\end{enumerate}

\section{Hypotheses}

We believe a variety of topic model visualizations will offer assistance in solving the
proposed problems. Using topic models, mental health conversations may be
summarized by a combination of topics, such as job-related issues, family troubles,
relationship difficulties, or self-injurious behavior, to name a few. Topic models also
provide indexing information, which tells us where each specific topic can be found
in the conversation text.

\subsection{Context Switching}

As the medium of texting usually involves gaps in response time during a
conversation, counselors often switch context between talking to different clients.
When a counselor returns to a previous conversation after a client response,
he or she may have to spend time recalling what that particular conversation was
about. However, if the counselor was given a visual summary of the conversation,
with the option of quickly reading through chat details, less time may be spent
recognizing the conversation topics. This approach can minimize both the time
a counselor spends not talking to a client and the cognitive load that context
switching has on the counselor.

\subsection{Shift Changes}

Counselors usually handle incoming client texts in shifts. A shift change may occur
in the middle of conversations, in which case the leaving counselor gives the
incoming counselor a brief summary of the talking points so he or she can take
over. However, this summary is general and transient, and the incoming counselor
would have to take time scanning through the existing conversation text for details.
We suspect that a permanent visual summary computed using topic models would be
more helpful for the incoming counselor. The visualizations can provide different
levels of detail depending on what the counselor needs to know about the
conversation history. Visual indexing can quickly point him or her to the parts of
the conversation related to a certain topic. This technique minimizes the amount of
time necessary to search through the text for details and potentially improves the
quality of client service by better preparing the new counselor for the interaction.

\subsection{Automated Reports}

Certain crisis organizations require their counselors to complete reports on their
conversations with clients. Although
these reports may be helpful to some, they can be time-consuming to manually
fill out. That time could be better spent interacting with clients. Given that we can
use algorithms to analyze conversation text and extract information, we believe
that this information can be used to automatically pre-populate reports.
Another idea is to use visual summaries as a complementary form of a report.

\subsection{Conversation Trends}

As previously mentioned, counselors must keep track of multiple conversations
at a time. These conversations may also contain gaps of time due to the use of
texting. In order to aid the counselor's memory, we believe that displaying topic
trends over time for each conversation could be useful. Showing trends, including
where topics appear in the course of a conversation and how they accumulate, may
potentially improve the quality of client service. A chart of topic trends could alert the
counselor to important focus points. For example, if the topic of self-injurious behavior
is on the rise, the counselor might want to react in a certain way to prevent escalation
of injury. Conversation trends may also be useful for organization leaders to detect
patterns that might be of use in supporting clients.

\section{Contributions}

Based on a topic model developed from a collection of real mental health
conversations, I designed and implemented a website prototype for Crisis Text
Line with four visualizations. These visualizations were designed based on four
different levels of granularity, so the counselor can choose the amount of detail
he or she wants.

The \textbf{Topic List} visualization lists the topics discovered in a conversation that
are above a certain threshold. Topics are ordered from highest to lowest percentage
detected in the conversation. This visualization is a quick, glance-able summary of the
conversation topics.

The \textbf{Donut Chart} visualization adds a small level of detail by displaying the
topic proportions in a pie chart variation to show the parts of the whole relationship.
User interaction by hovering over the chart or the legend provides the topic
percentages for quantitative information.

The \textbf{Line Chart} visualization reaches finer granularity by revealing topic
proportions at the message level, where each client message in the conversation
is analyzed for topics. A line exists for each topic above a certain threshold that
shows the trend of that topic throughout the conversation timeline. There is also
the option of viewing the accumulation of topic proportions across the conversation.
When the user clicks on a topic, points are displayed to reveal the client messages
in the conversation that contain the topic.

The \textbf{Scatter Plot} visualization is the deepest detail level, allowing the user to
click on the topic instances that occur throughout the conversation. The conversation
text then automatically scrolls to the appropriate message. The size of the scatter plot
points represent the proportion of that topic in the corresponding message.

\section{Thesis Outline}

Chapter two presents related work, consisting of topic models, visualizations of
topic models for other fields, and mental health topic modeling. The scope and
limitations of the thesis is also included in this chapter.

Chapter three describes the contextual inquiry done with crisis counselors to
analyze the needs of our users for a better design.

Chapter four discusses the design of the four visualizations contributed in this
this thesis: a topic list, a donut chart, a multi-series line chart, and a scatter plot.

Chapter five explains how the system was implemented and lists the existing
technologies that were used.

Chapter six evaluates the visualizations based on the user test results.

Chapter seven explores ideas for future work, some of which could not be completed
due to time, resource, and technological constraints.

Finally, chapter eight discusses the main contributions presented in this thesis.
