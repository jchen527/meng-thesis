% $Log: abstract.tex,v $
% Revision 1.1  93/05/14  14:56:25  starflt
% Initial revision
% 
% Revision 1.1  90/05/04  10:41:01  lwvanels
% Initial revision
% 
%
%% The text of your abstract and nothing else (other than comments) goes here.
%% It will be single-spaced and the rest of the text that is supposed to go on
%% the abstract page will be generated by the abstractpage environment.  This
%% file should be \input (not \include 'd) from cover.tex.
Crisis Text Line supports many people with mental health issues through texting. Unfortunately, this support is limited by the number of counselors and the time each counselor volunteers, as well as the cognitive load needed to manage multiple conversations at once for long periods of time. We conducted a contextual inquiry with crisis counselors to find the specific problems in their workflow. In order to maximize the time and brainpower counselors spend helping clients, we believe topic modeling can provide summaries of conversation text to aid management. Four simple and familiar visualizations were developed to present the topic model data. Counselors can choose from varying levels of granularity: 1) a list of conversation topics, 2) a pie chart of topic percentages, 3) a line chart of topic trends throughout a conversation, and 4) a scatter plot of specific locations in the text where the topics were detected. Our hypothesis is that these visualizations will help counselors keep track of different conversations, provide clarifying details, and improve the quality of client support. Finally, the visualizations were evaluated through a user study with crisis counselors to determine their effectiveness against a control interface.
