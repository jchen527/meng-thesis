\chapter{Related Work and Scope}

In this section, we first summarize relevant research presented in three categories: topic modeling, topic model visualizations, and general visualization techniques. We then provide the scope and limitations of this thesis project.

\section{Topic Modeling}

Probabilistic topic models \cite{blei-topicmodel} are algorithms that aim to extract the main themes from a large collection of documents. These algorithms use statistics to analyze the words in each document's text and organize them into topics. Topic modeling can be used to aid summarization and information retrieval for various types of data without the need for humans to manually annotate a large amount of text.

The simplest topic model is \textit{latent Dirichlet allocation} (LDA) \cite{blei-topicmodel}. LDA uses a statistical process to discover the topics in a corpus of documents. A \textit{topic} is formally defined as a distribution over a fixed vocabulary. For example, a \textit{genetics} topic should have the words \textit{genetics} and \textit{genes} with high probability. LDA consists of reverse-engineering an imaginary generative process. This process begins by taking a random distribution over topics. Each word for each document is then generated by randomly choosing a topic from the distribution over topics and randomly choosing a word from that topic's distribution over words. We refer to the topics, the per-document topic distributions, and the per-document per-word topic assignments as the topic structure. This generative process must be reverse-engineered because the words in the documents are observed, while the hidden topic structure that most likely generated the words must be inferred.

The topic modeling algorithm used for the visualizations in this thesis is contributed by Karthik Dinakar. His approach is similar to the algorithm developed for a previous story-matching project \cite{dinakar-mtv} that involves mental health topic models, which is a very new research area. First, LDA is applied to the set of documents. The output is topics, in the form of word clusters, and a distribution over the topics for each document. Each word cluster is then analyzed by a human and interpreted as a theme if possible. This process iterates with an increasing number of desired topics until a satisfactory collection of themes have been extracted from the documents. Each document has a distribution over the themes. The topic modeling used in this thesis is slightly different in two ways. First, the documents are conversations between a client and a counselor, so only the words in the client messages are analyzed. Second, each client message also has a distribution over themes, so the conversations can be broken down further and annotated with themes.

We will not discuss the specifics of topic modeling at the deeper level of probability and statistics because this thesis is concentrated on visualization. The purpose of this overview is to familiarize the reader with the concept of topic modeling, focusing on how it is used to extract a set of topics from a document corpus and annotate documents with topics based on the document words.

\section{Visualizing Topic Models}

Numerous research projects in topic model visualization revolve around visualizing documents to show relationships or similarities based on their latent topics. \textit{Probabilistic Latent Semantic Visualization} (PLSV) \cite{plsv} is a topic model approach to visualizing documents and topics as coordinate points in a visualization space. The distances between documents and topics are based on the topic distribution of a document. \textit{Topic maps} \cite{topic-maps} and \textit{Exemplar-based Visualization} (EV) \cite{ev} provide similar graphs of a large collection of documents, with document points color-coded by their dominant topic. The Stanford Dissertation Browser \cite{interpretation-trust} is also a notable visualization developed to evaluate word and topic similarities between the Ph.D. theses of different departments over time. The general purpose of these visualizations is to show documents with similar topics in clustered areas for a global overview of the corpus.

Now we turn to a few systems that are more relevant to our research in terms of their goals, end-users, or visual design. We are focused on summarizing individual documents using topics, revealing topic trends of a document over time, and indexing topics within document text using simple visualizations for non-technical users. Our developed visualizations were inspired by different aspects of these projects.

The Wikipedia navigator \cite{wikipedia} was specifically designed to summarize the corpus and show relationships between textual content and topics for non-technical users. Three straightforward visualizations were produced: an overview page that lists the set of topics associated with all documents, a topic page that displays associated words as well as related document and topic links, and a document page showing the content in addition to related document links and a pie chart of related topics. These visuals allow the user to be completely unaware of the underlying LDA topic models.

The interactive visual text analysis tool TIARA \cite{tiara} summarizes a corpus over time using a stream graph with topic layers and distributed keywords. ThemeRiver \cite{theme-river} provides the same type of graph without keywords. The height of the topic layer areas illustrate the strength of each topic at a certain point in time. Although I personally find stream graphs difficult to comprehend, these visualizations show that area or line charts can be useful for expressing topic trends over time.

Finally, Termite \cite{termite} is a visual analysis tool for evaluating the quality of topic models. The main visualization of this tool is a term-topic matrix that can be described as a scatter plot of words for each topic, with the size of each point proportional to the word frequency for that topic. Clicking on a topic in this matrix shows its representative documents and a one-dimensional plot of where topical terms can be found within each document. These simple designs seem effective for visually indexing topics in each document.

\section{Visualization Techniques}



\section{Scope and Limitations}


