\chapter{Related Work and Scope}

In this section, we first summarize relevant research presented in three categories: topic modeling, topic model visualizations, and general visualization techniques. We then provide the scope and limitations of this thesis project.

\section{Topic Modeling}

Probabilistic topic models \cite{blei-topicmodel} are algorithms that aim to extract the main themes from a large collection of documents. These algorithms use statistics to analyze the words in each document's text and organize them into topics. Topic modeling can be used to aid summarization and information retrieval for various types of data without the need for humans to manually annotate a large amount of text.

The simplest topic model is \textit{latent Dirichlet allocation} (LDA) \cite{blei-topicmodel}. LDA uses a statistical process to discover the topics in a set of documents. A \textit{topic} is formally defined as a distribution over a fixed vocabulary. For example, a \textit{genetics} topic should have the words \textit{genetics} and \textit{genes} with high probability. LDA consists of reverse-engineering an imaginary generative process. This process begins by taking a random distribution over topics. Each word for each document is then generated by randomly choosing a topic from the distribution over topics and randomly choosing a word from that topic's distribution over words. We refer to the topics, the per-document topic distributions, and the per-document per-word topic assignments as the topic structure. This generative process must be reverse-engineered because the words in the documents are observed, while the hidden topic structure that most likely generated the words must be inferred.

The topic modeling algorithm used for the visualizations in this thesis is contributed by Karthik Dinakar. His approach is similar to the algorithm developed for a previous story-matching project \cite{dinakar-mtv} that involves mental health topic models, which is a very new research area. First, LDA is applied to the collection of documents. The output is topics, in the form of word clusters, and a distribution over the topics for each document. Each word cluster is then analyzed by a human and interpreted as a theme if possible. This process iterates with an increasing number of desired topics until a satisfactory set of themes have been extracted from the documents. Each document has a distribution over the themes. The topic modeling used in this thesis is slightly different in two ways. First, the documents are conversations between a client and a counselor, so only the words in the client messages are analyzed. Second, each client message also has a distribution over themes, so the conversations can be broken down further and annotated with themes.

We will not discuss the specifics of topic modeling at the deeper level of probability and statistics because this thesis is focused on visualization. The purpose of this overview is to familiarize the reader with the concept of topic modeling, focusing on how it is used to extract a set of topics from a document collection and annotate documents with topics based on the document words.

\section{Visualizing Topic Models}



\section{Visualization Techniques}



\section{Scope and Limitations}


