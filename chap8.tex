\chapter{Conclusion}

Crisis hotlines are a great channel for mental health counseling. In order to prevent
tragic events, counselors must handle as many clients as possible, especially those in
high-risk situations, without overloading themselves. Providing assistive tools is the
least we can do for these counselors who save human lives on a regular basis. Our
focus is to help them by increasing efficiency and decreasing burnout.

In this thesis, we performed a contextual inquiry on crisis counselors. The
results allowed us to target our system toward reducing the amount of note-taking
and transcript-reading that counselors do. We combined two effective technical
approaches, topic modeling and graphical visualization, that are traditionally used to
tackle different problems to design four topic model visualizations. These real-time
visualizations summarize the conversation topics and index the messages in a
conversation dynamically using data from the LDA machine learning algorithm. We
developed a fully functional system that receives text messages from clients and lets
the counselor offer a chat response. Finally, we evaluated the visualizations in the
context of the counselor workflow as it was observed in the contextual inquiry.

Based on the feedback we collected, we can confidently conclude that aspects of
our system are useful for crisis counselors. It was extremely encouraging to see the
enthusiasm that real counselors had for our technology. We determined that visual
indexing is a valuable feature for filtering specific messages from a conversation
transcript. However, there is definitely more work to be done to extend the effectiveness
of our interface to more use cases.

Throughout the design iterations for the topic model visualizations, we also
experienced how nontrivial it is to develop an interface that is centered around a specific
user. It can be difficult to reject more technical and complex ideas as a computer
scientist in favor of simple designs that do not seem as impressive. Of course, it does
not matter how the developer would use the interface, only whether it serves the right
function for end-users.

Overall, I am grateful to have been involved in a project that has the potential
for high societal impact.  This thesis has proven to me that technical solutions can
be effective even in less computational fields such as mental health, where feelings
and emotions are dominant. I sincerely hope that others will continue this important
work and expand this new research area.
